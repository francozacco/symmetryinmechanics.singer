\documentclass[11pt]{article}
\usepackage{amssymb}
\usepackage{amsthm}
\usepackage{enumitem}
\usepackage{physics,amsmath}
\usepackage{bm}
\usepackage{adjustbox}
\usepackage{mathrsfs}
\usepackage{graphicx}
\usepackage{siunitx}
\usepackage[mathscr]{euscript}


\title{\textbf{Solved selected problems of Symmetry in Mechanics by Stephanie Singer}}
\author{Franco Zacco}
\date{}

\addtolength{\topmargin}{-3cm}
\addtolength{\textheight}{3cm}

\newcommand{\R}{\mathbb{R}}
\newcommand{\hatr}{\bm{\hat{r}}}
\newcommand{\hatn}{\bm{\hat{n}}}
\newcommand{\hatx}{\bm{\hat{x}}}
\newcommand{\haty}{\bm{\hat{y}}}
\newcommand{\hatz}{\bm{\hat{z}}}
\newcommand{\hatth}{\bm{\hat{\theta}}}
\newcommand{\hatphi}{\bm{\hat{\phi}}}
\newcommand{\hatrho}{\bm{\hat{\rho}}}
\newcommand{\er}{\bm{e}_r}
\newcommand{\etht}{\bm{e}_\theta}
\newcommand{\uvi}{\bm{i}}
\newcommand{\uvj}{\bm{j}}
\newcommand{\uvk}{\bm{k}}

\theoremstyle{definition}
\newtheorem*{solution*}{Solution}
\renewcommand*{\proofname}{Solution}

\begin{document}
\maketitle
\thispagestyle{empty}

\section*{Chapter 1 - The Two-Body Problem}

\begin{proof}{\textbf{Exercise 6}}
\begin{itemize}
    \item [1.] Given $\bm{f},\bm{g}:\R \to \R^3$ we know that the cross product
    between $\bm{f}(t)$ and $\bm{g}(t)$ give us
    \begin{align*}
        (\bm{f}(t) \times \bm{g}(t))_1 &= (f_2g_3 - f_3g_2)\\
        (\bm{f}(t) \times \bm{g}(t))_2 &= (f_3g_1 - f_1g_3)\\
        (\bm{f}(t) \times \bm{g}(t))_3 &= (f_1g_2 - f_2g_1)
    \end{align*}
    Also we know that 
    \begin{align*}
        \frac{d}{dt}(\bm{f}(t) \times \bm{g}(t))
        &= \bigg(
            \frac{d}{dt}(\bm{f}(t) \times \bm{g}(t))_1,
            \frac{d}{dt}(\bm{f}(t) \times \bm{g}(t))_2,
            \frac{d}{dt}(\bm{f}(t) \times \bm{g}(t))_3
        \bigg)
    \end{align*}
    Then applying the product rule we get that
    \begin{align*}
        \frac{d}{dt}(\bm{f}(t) \times \bm{g}(t))_1 &= 
        (f_2'g_3 + f_2g_3' - f_3'g_2 - f_3g_2')\\
        \frac{d}{dt}(\bm{f}(t) \times \bm{g}(t))_2 &= 
        (f_3'g_1 + f_3g_1' - f_1'g_3 - f_1g_3')\\
        \frac{d}{dt}(\bm{f}(t) \times \bm{g}(t))_3 &= 
        (f_1'g_2 + f_1g_2' - f_2'g_1 - f_2g_1')
    \end{align*}
    On the other hand, we know that
    \begin{align*}
        (\bm{f}'(t)\times \bm{g}(t))_1 = (f_2'g_3 - f_3'g_2)\\
        (\bm{f}'(t)\times \bm{g}(t))_2 = (f_3'g_1 - f_1'g_3)\\
        (\bm{f}'(t)\times \bm{g}(t))_3 = (f_1'g_2 - f_2'g_1)
    \end{align*}
    And that
    \begin{align*}
        (\bm{f}(t)\times \bm{g}'(t))_1 = (f_2g_3' - f_3g_2')\\
        (\bm{f}(t)\times \bm{g}'(t))_2 = (f_3g_1' - f_1g_3')\\
        (\bm{f}(t)\times \bm{g}'(t))_3 = (f_1g_2' - f_2g_1')
    \end{align*}
    So 
    \begin{align*}
        (\bm{f}'(t)\times \bm{g}(t))_1 + (\bm{f}(t)\times \bm{g}'(t))_1
        = (f_2'g_3 - f_3'g_2) + (f_2g_3' - f_3g_2')\\
        (\bm{f}'(t)\times \bm{g}(t))_2 + (\bm{f}(t)\times \bm{g}'(t))_2
        = (f_3'g_1 - f_1'g_3) + (f_3g_1' - f_1g_3')\\
        (\bm{f}'(t)\times \bm{g}(t))_3 + (\bm{f}(t)\times \bm{g}'(t))_3
        = (f_1'g_2 - f_2'g_1) + (f_1g_2' - f_2g_1')
    \end{align*}
    Hence, we see that
    \begin{align*}
        \frac{d}{dt}(\bm{f}(t) \times \bm{g}(t))_1 &= 
        (\bm{f}'(t)\times \bm{g}(t))_1 + (\bm{f}(t)\times \bm{g}'(t))_1\\
        \frac{d}{dt}(\bm{f}(t) \times \bm{g}(t))_2 &= 
        (\bm{f}'(t)\times \bm{g}(t))_2 + (\bm{f}(t)\times \bm{g}'(t))_2\\
        \frac{d}{dt}(\bm{f}(t) \times \bm{g}(t))_3 &= 
        (\bm{f}'(t)\times \bm{g}(t))_3 + (\bm{f}(t)\times \bm{g}'(t))_3
    \end{align*}
    This implies that
    \begin{align*}
        \frac{d}{dt}(\bm{f}(t) \times \bm{g}(t)) &= 
       \bm{f}'(t)\times \bm{g}(t) + (\bm{f}(t)\times \bm{g}'(t))
    \end{align*}
    \item [2.]
    The dot product between $\bm{f}(t)$ and $\bm{g}(t)$ gives us
    \begin{align*}
        \bm{f}(t) \cdot \bm{g}(t) = f_1g_1 + f_2g_2 + f_3g_3
    \end{align*}
    Hence the derivative with respect to $t$ is
    \begin{align*}
        \frac{d}{dt}(\bm{f}(t) \cdot \bm{g}(t))
        &= f_1'g_1 + f_1g_1' + f_2'g_2 + f_2g_2' + f_3'g_3 + f_3g_3'
    \end{align*}
    But also we see that 
    \begin{align*}
        \bm{f}'(t) \cdot \bm{g}(t) + \bm{f}(t) \cdot \bm{g}'(t)
        = (f_1'g_1 + f_2'g_2 + f_3'g_3) + (f_1g_1' + f_2g_2' + f_3g_3')
    \end{align*}
    Therefore
    \begin{align*}
        \frac{d}{dt}(\bm{f}(t) \cdot \bm{g}(t))
        = \bm{f}'(t) \cdot \bm{g}(t) + \bm{f}(t) \cdot \bm{g}'(t)
    \end{align*}
    \item [3.]
    Let $\bm{v},\bm{w} \in \R^3$ and suppose we orientate the axes of
    $\R^3$ such that $\bm{v}$ and $\bm{w}$ lay on the $xy$ plane on this
    system of coordinates. We also rotate the $xy$ axes such that the $x$
    axis match the direction of $\bm{v}$ then vectors have the following 
    coordinates
    \begin{align*}
        \bm{v} = \begin{pmatrix}
            v_x\\ 0\\ 0
        \end{pmatrix}
        \quad\quad
        \bm{w} = \begin{pmatrix}
            w_x\\ w_y\\ 0
        \end{pmatrix}
    \end{align*}
    So we see that the area $A$ of the parallelogram spanned by $\bm v$ and
    $\bm w$ is $A = v_x w_y$.

    On the other hand, computing the cross product between $\bm{v}$ and $\bm{w}$
    gives us
    \begin{align*}
        \bm{v} \times \bm{w} &= 
        (v_yw_z - v_zw_y)\uvi + (v_zw_x - v_xw_z)\uvj + (v_xw_y - v_yw_x)\uvk\\
        &= (v_xw_y) \uvk
    \end{align*}
    Where $\uvi, \uvj$ and $\uvk$ are unit vectors in the direction of $x,y$
    and $z$ respectively. 

    Therefore $|\bm{v} \times \bm{w}| = \sqrt{(v_xw_y)^2} = v_xw_y$
    which is the area the parallelogram spanned by $\bm v$ and $\bm w$.
\end{itemize}
\end{proof}
\cleardoublepage
\begin{proof}{\textbf{Exercise 7}}
    We know that
    \begin{align*}
        r_x &= \rho\cos\theta\\
        r_y &= \rho\sin\theta
    \end{align*}
    And that
    \begin{align*}
        \frac{p_x}{\mu} &= \dot\rho\cos\theta - \rho\dot\theta\sin\theta\\
        \frac{p_y}{\mu} &= \dot\rho\sin\theta + \rho\dot\theta\cos\theta
    \end{align*}
    Also, the equation of motion states that
    $$\dv{\bm{p}}{t} = -\frac{GM\mu}{|\bm{r}|^3}\bm{r}$$
    Let us compute $\frac{1}{\mu}\dv{\bm{p}}{t}$ as follows
    \begin{align*}
        \frac{1}{\mu}\dv{p_x}{t} &= \dv{\dot\rho}{t}\cos\theta
        - 2\dot\rho\dot\theta\sin\theta
        - \rho\left(\dv{\dot\theta}{t}\sin\theta + \dot\theta^2\cos\theta\right)\\
        \frac{1}{\mu}\dv{p_y}{t} &= \dv{\dot\rho}{t}\sin\theta
        + 2\dot\rho\dot\theta\cos\theta
        + \rho\left(\dv{\dot\theta}{t}\cos\theta - \dot\theta^2\sin\theta\right)
    \end{align*}
    On the other hand, we see that
    $|\bm{r}| = \sqrt{\rho^2(\cos^2\theta + \sin^2\theta)} = \rho$
    hence
    \begin{align*}
        -\frac{GM}{\rho^3}r_x &= -\frac{GM}{\rho^2}\cos\theta\\
        -\frac{GM}{\rho^3}r_y &= -\frac{GM}{\rho^2}\sin\theta
    \end{align*}
    Then the equations of motion become
    \begin{align*}
        \dv{\dot\rho}{t}\cos\theta
        - 2\dot\rho\dot\theta\sin\theta
        - \rho\left(\dv{\dot\theta}{t}\sin\theta + \dot\theta^2\cos\theta\right)
        &= -\frac{GM}{\rho^2}\cos\theta\\
        \dv{\dot\rho}{t}\sin\theta
        + 2\dot\rho\dot\theta\cos\theta
        + \rho\left(\dv{\dot\theta}{t}\cos\theta - \dot\theta^2\sin\theta\right)
        &= -\frac{GM}{\rho^2}\sin\theta
    \end{align*}
    Multiplying the first equation by $\cos\theta$, the second one by
    $\sin\theta$ and then summing them we get that
    \begin{align*}
        \dv{\dot\rho}{t}~(\cos^2\theta + \sin^2\theta)
        - \rho\dot\theta^2(\cos^2\theta + \sin^2\theta)
       &= -\frac{GM}{\rho^2}(\cos^2\theta + \sin^2\theta)\\
       \dv{\dot\rho}{t} - \rho\dot\theta^2 + \frac{GM}{\rho^2} &= 0
   \end{align*}
    Which is one of the equations we are looking for.
    
    Now, if we multiply the first equation by $\sin\theta$, the second one by
    $\cos\theta$ and then we subtract the second from the first one we get that
    \begin{align*}
        - 2\dot\rho\dot\theta(\sin^2\theta + \cos^2\theta)
        - \rho\dv{\dot\theta}{t}~(\sin^2\theta + \cos^2\theta)
        &= 0\\
        2\dot\rho\dot\theta + \rho\dv{\dot\theta}{t} &= 0
    \end{align*}
    Which is another equation that we are looking for.

    Finally by definition we know that
    \begin{align*}
        \dv{\rho}{t} &= \dot\rho\qquad
        \dv{\theta}{t} &= \dot\theta
    \end{align*}
    Which completes the last two equations we are looking for.
\end{proof}
\cleardoublepage
\begin{proof}{\textbf{Exercise 8}}
    We know that $q = 1/\rho$ hence $\rho = 1/q$ so derivating we get that
    \begin{align*}
        \dv{\rho}{t} &= -\frac{1}{q^2}\dv{q}{t}\\
        \dv{\rho}{t} &= -\frac{1}{q^2}\dv{q}{\theta}\dv{\theta}{t}
    \end{align*}
    And substituting the equation for $d\theta/dt$ we have from the hint
    we get that
    \begin{align*}
        \dv{\rho}{t}
        &= -\frac{1}{q^2}\dv{q}{\theta}\left(q^2\frac{|\tilde{\bm{L}}|}{\mu}\right)\\
        &=-\frac{|\tilde{\bm{L}}|}{\mu}\dv{q}{\theta}
    \end{align*}
    Now we derivate with respect to $t$ again
    \begin{align*}
        \dv[2]{\rho}{t}
        &=-\frac{|\tilde{\bm{L}}|}{\mu}\dv{t}(\dv{q}{\theta})\\
        &=-\frac{|\tilde{\bm{L}}|}{\mu}\dv[2]{q}{\theta}\dv{\theta}{t}
    \end{align*}
    Finally, we replace again the value we have for $d\theta/dt$
    \begin{align*}
        \dv[2]{\rho}{t}
        &=-\frac{|\tilde{\bm{L}}|^2}{\mu^2}q^2\dv[2]{q}{\theta}
    \end{align*}
\end{proof}
\cleardoublepage
\begin{proof}{\textbf{Exercise 9}}
    No, Mars will not crash into the sun because for this to happen the
    ellipse path that Mars is following needs to transform into an hyperbola
    and we saw that angular momentum is conserved hence, a planet cannot slow
    down too much, ruling out bounded motion along a parabola or hyperbola.
\end{proof}
\cleardoublepage
\begin{proof}{\textbf{Exercise 10}}
We know that the polar coordinate $\rho$ from the sun to the comet Hale-Bopp
is
\begin{align*}
    \frac{1}{\rho} = \frac{GM\mu^2}{|\bm{\tilde{L}}|^2} + A\cos(\theta + \theta_0)
\end{align*}
Also, we know the angular momentum is given by
\begin{align*}
    |\bm{\tilde{L}}| = \mu\rho^2\dot\theta = \mu\rho v
\end{align*}
Where $v$ is the tangential velocity and since the angular momentum is
conserved, for the perihelion we have that
\begin{align*}
    \frac{|\bm{\tilde{L}}|}{\mu} = \rho_p v_p = (1.36732\times 10^8)(44)
    = 6016208000~\frac{km^2}{s}
\end{align*}
Considering we are at the perihelion we can take $\theta + \theta_0 = 2\pi$
and hence $A$ is 
\begin{align*}
    \frac{1}{\rho_p} &= \frac{GM}{\rho_p^2v_p^2} + A\\
    A &= \frac{1}{\rho_p} - \frac{GM}{\rho_p^2v_p^2}
\end{align*}
Then given that $GM = 132712440044~km^3/s^2$ we have that
\begin{align*}
    A &= \frac{1}{1.36732\times 10^8}
    - \frac{132712440044}{(6016208000)^2}\\
    A &= 3.64695\times 10^{-9}~km^{-1}
\end{align*}
Now, let us suppose we are at the aphelion then must be that
$\cos\theta + \theta_0 = -1$ hence
\begin{align*}
    \frac{1}{\rho_a} &= \frac{GM\mu^2}{|\bm{\tilde{L}}|^2} - A\\
    &= \frac{132712440044}{(6016208000)^2}
    - 3.64695\times 10^{-9}\\
    &= 1.96638\times 10^{-11}
\end{align*}
Then $\rho_a = 50854756626$ and the semi-major axis is given by
\begin{align*}
    a = \frac{\rho_a + \rho_p}{2} = 25495744313~km \approx 170 AU
\end{align*}
Finally, from Kepler's third law the period of the Hale-Bopp comet is
\begin{align*}
    \tau^2 &= \frac{4\pi^2 a^3}{GM_s}\\
    \tau^2
    &= \frac{4\pi^2 (25495744313)^3}{32712440044}\\
    \tau &= \sqrt{2.00009 \times 10^{22}~s^2}\\
    \tau &= 141424594179~s \approx 4484~\text{years}
\end{align*}
Therefore Hale-Bopp will return in $4484~\text{years}$.
\end{proof}
\cleardoublepage
\begin{proof}{\textbf{Exercise 11}}
    We know that
    \begin{align*}
        \frac{1}{(b^2/a)} = \frac{GM\mu^2}{|\tilde{\bm{L}}|}
    \end{align*}
    But since we are considering a circular orbit then $b=a$ and 
    $|\tilde{\bm{L}}| = \mu\rho v= \mu a v$ where $v$ is the velocity of
    the satellite then
    \begin{align*}
        \frac{1}{a} &= \frac{GM\mu^2}{\mu^2 a^2 v^2}\\
        v^2 &= \frac{GM}{a}\\
        v &= \sqrt{\frac{GM}{a}}
    \end{align*}
    Now, to put the satellite into a circular orbit that passes over the north
    and south poles the velocity direction needs to change $\pi/2$ radians.
    Then the change in velocity $\Delta v$ is the hypotenuse of a right triangle
    where each side is $v$ then
    \begin{align*}
        \Delta v = \sqrt{v^2 + v^2} = \sqrt{2}v
    \end{align*}
\end{proof}
\end{document}